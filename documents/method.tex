Before any actual work could start, one preliminary goal was to figure out how we could make our group work together as one. Actually this challenge is even more challenging in this project than in a normal work situation: No organization is in order, no predefined roles, no actual project goals and the likes. This chapter will focus on these challenges and how we tried to handle these. We will highlight different methods to create social interaction and understanding. We will focus on how one can rationalize collaboration. Afterward we will discuss the different tools we used throughout the project life cycle with collaboration in mind. Finally, we will discuss how one can manage a virtual project.

\section{Social Context}
When we discuss the \textit{Social Context}, we discuss the direct milieu in which the person is and how different factors can influence this person. Communication is also a part of the social context, which is not necessarily only between two persons but can be between one to many persons, in different time zones, different cultures etc. 
\begin{itemize}
  \item Common ground
  \item Coupling of work
  \item Trust
	\item Ethnocentrism
	\item Collaboration readiness
	\item Technology readiness
	\item First impression matters
	\item Communication
\end{itemize}

\section{Collaborative Work}
\begin{itemize}
  \item Cooperative work
  \item Articulation work
  \item Awareness
	\item Coordination of actions
	\item Management of coupling
	\item Coordination of mechanisms
\end{itemize}

\section{Groupware Technologies}
\begin{itemize}
  \item Adoption process
  \item Adaptation
  \item Critical mass
\end{itemize}

\section{Virtual Project Management}
\begin{itemize}
  \item Continuities
  \item Discontinuities
  \item Virtual meetings
	\item Virtual team dynamics
\end{itemize}