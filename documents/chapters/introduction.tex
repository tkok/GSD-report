\section{Introduction} \label{sec:introduction}
The research field of building and home automation experiences a lot of growth - not least because of the promise to reduce energy consumption by more intelligent control, but also due to the possibility of heightened human comfort. Constructing resource efficient buildings makes sense, both in a political and economical perspective. In \cite{janssen2004towards} it is stated that residential buildings use about 82\% of the total energy consumption on space heating and water heating. Electric appliances uses 11\%. Manually controlling an energy-saving policy is
a time-consuming and inefficient task and the user have to know a lot of different equipment as well as information about the building. It is, however, achievable using building automation. 

Regulating the environment of buildings to heighten human comfort, without annoying or irritating them \cite{futurehome} also requires that building components can communicate and be controlled in a fashion that seems smart or intelligent by humans. Buildings today might come equipped with a suite of sensors and actuators, opening up for a degree of customizable control, and our collective need is that buildings can adapt to the users and the sensor-perceived environment. We define a building automation program as a policy, and the software entity controlling policies we define as the policy engine.

Policies can be based on semi-static data, like time and weekdays. However this can have unforeseen and unwanted consequences. For example, a policy governing lightning activated merely by a static time schedule, might entail problems for people attending a rarely occurring late-night party in the building. If the event calendar of the building is accessible to the policy engine, a conditional event-checking statement might ensure continuous lighting. However, in order to achieve a more fine grained control, sensory input is needed. We define the task of interacting with these policies, as residing with Facility Management (FM). 

By employing a policy engine, with access to the buildings sensors and actuators, both the building owner, the users and the administrators of the building benefits from the automation provided. If policies are correctly defined, building owners save energy and natural resources while providing extra comfort to their tenants REF NEEDED. Building users can experience a building autonomously adjusting its internal environment to suit their comfort and needs, while FM can achieve fine-grained control of the building.

This project was defined in the course Global Software Development at ITU. Our global development team consists of 5 students from IT-University of Copenhagen and 4 from Strathmore University, Nairobi Kenya. We have been provided a Building Simulator, making up for the lack of a real building. The complexity with integrating to hardware and communication protocols are therefore avoided, however the simulator has provided other complexities as will be evident later in this paper. The development focus of this project is therefore geared towards this simulator. The end product is a web-based management console, that allow for centralized flow control between sensors and actuators in the simulated building. This is achieved by implementing the policy engine that allows for automated actuator responses based on sensor feedback. An example at this would be closing the blinds in excess sunlight or turning of the heater when the windows are open.

In this paper we will; 
\begin{enumerate}
	\item distill requirements from course provided material and a non-exhaustive literature search on policy engines
	\item develop a software solution that implements these requirements.
	\item document the collaborative project between the IT University of Copenhagen, Denmark and Strathmore University, Nairobi Kenya.
\end{enumerate}

\section{Context}

\section{Problem}
