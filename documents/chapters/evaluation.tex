%\section{Evaluation}
%%%% ASLAK: Start by explaining why these areas are: 1. important 2. covers what is important
In our project we have two areas that we have been evaluating. The first being if our policy engine is actually working and enforcing the defined policies. The other being the front end interface, testing to uncover potential usability issues that needed solving.

\section{Policy Engine System Evaluation}

\subsection{Log Testing}

Log every action and check timestamps, values, actions etc...

\subsection{JUnit Testing}
.......

\section{Usability Test}

\subsection{Think Aloud Test}

In a thinking aloud test, you ask test participants to use the system while continuously thinking out loud — that is, simply verbalizing their thoughts as they move through the user interface.

To run a basic thinking aloud usability study, you need to do only 3 things:

Recruit representative users.
Give them representative tasks to perform. %%ASLAK: Perhaps someone from FM??
Shut up and let the users do the talking.


Cheap, simple, flexible....

Research shows that testing 5 persons uncover 80\% of all usability problems...


REFERENCE
"Thinking aloud may be the single most valuable usability engineering method." 1993, Usability Engineering, Jacob Nielsen

\section{Technology}
In the beginning of the project we talked about the many technologies available for web development. Obviously there are many approaches to the same solution. Focussing on working as a team, that all have different background experiences we decided to individually write down our development qualifications.

From all of the qualifications in the group, we joined together and made a top three technology approached (see Tabel \ref{tbl:dev_environment}. Every team member voted 1, 2 and 3 on an development approach. 1 being the most preferred, and three the less preferred. 

\begin{table}[h]
\caption{Team members preferred development approaches}\label{tbl:dev_environment}
    \begin{tabular}{rccccc}
    Dev environment        & Kasper & Thomas & Stefan & Rasmus & Nicolas \\
    PHP, JavaScript, MySQL & 2      & 1      & 3      & 3      & 1       \\
    C\#.Net, MS SQL         & 3      & 2      & 2      & 1      & 1       \\
    Java, JSP, Tomcat      & 1      & 3      & 1      & 2      & 3       \\
    \end{tabular}
\end{table}

One of the challenges in this project was technology readiness. Every team member has a preferred development language, which ultimately led to a long debate on settling with a development language, and what kind of framework the group should use.
The problem was that: either way - some team members would have a harder time participating with implementation than others.

Also some members had already lots of experience in development web applications, while other had very little. Not everyone was familiar with MVC based development, which for those unfamiliar became a challenge. The overall problem was simply having different level of knowledge regarding web development and technologies.

Ultimately however the group decided to go with Java as the main programming language, along with Javascript for front-end parts.


Various milestone for the project was planned and the different implementation tasks, was assigned to different team members. Most noticeable the team unintentionally created subgroups: divided in team members working with front-end tasks, and team members working with backend tasks.
However, frequent meetings and online conversations led to positive awareness on what everyone was working with.

Furthermore the team utilize an online project management site, from which the team members could keep each other informed of work in progress.

Also Github was used to easily share and version code and documentation between team members. This made it possible for everyone to see changes and follow ongoing development.
