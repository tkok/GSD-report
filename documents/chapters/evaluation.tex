%\section{Evaluation}
%%%% ASLAK: Start by explaining why these areas are: 1. important 2. covers what is important
In our project we have two areas that we have been evaluating. The first being if our policy engine is actually working and enforcing the defined policies. The other being the front end interface, testing to uncover potential usability issues that needed solving.

\section{Policy Engine System Evaluation}
\label{policy-engine-system-evaluation}

\subsection{Log Testing}
\label{log-test}
Within the code of our software product we have used intensive logging. Doing so we have been able to verify the results and outcome while testing.

For quality control we have also been using these log-files. Verifying our solution by cross-referencing every action during tests, and match it with the actual data stored in the database.

\textbf{(Note: Insert an example from the log file)}

Log every action and check timestamps, values, actions etc...

\subsection{JUnit Testing}
To further test the software we made, we performed a range of JUnit Tests. JUnit is a unit testing framework for the Java programming language, which our software solution is based on. The unit test validate specified checks. 

In simple terms. Imagine you made a function that adds 2 \& 2. You know the outcome should be 4. So you make a JUnit test that verify the result using assertions. If the outcome compared is indeed 4, the JUnit test returns the boolean value \textit{true}, and \textit{false} if not.

Image however that the test is much larger, and much more complex the JUnit testing becomes very useful.

\section{Usability Test}
\label{sec:usability-test}
To evaluate on the usability of our policy engine we decided to make a test with candidates outside of our development group.

In general, when it come to "best practices of usability tests", the Think Aloud Protocol is considered one of the most valuable \cite{Nielsen1993}.

Therefore we decided to use it for our evaluation. Also the Think Aloud Protocol, is inexpensive in cost and time, and it is easy to set up and gives very valuable result from real-life scenarios. 

\subsection{Think Aloud Protocol}
In a thinking aloud test, you ask test participants to use the system while continuously thinking out loud — that is. Simply verbalizing their thoughts as they move through the user interface, and take actions.

To run a basic thinking aloud usability study, 3 things is required:
\begin{itemize}
\item Recruit representative users.
\item Inform them of representative tasks to perform. %%ASLAK: Perhaps someone from FM??
\item Avoid interference and let the users speak their actions.
\end{itemize}

We invited five people to a think aloud test. Research shows that having just five people will potentially uncover 80\% of all usability problems \footnote{URL: http://www.nngroup.com/articles/why-you-only-need-to-test-with-5-users/}.

The candidates is male and female students at itu, between 25 and 30. Some of them study software development and have a lot of knowledge in programming, others study digital design and knows more about human computer interaction and usability. However they all have web development in common.

\subsubsection{Tasks}
We arranged 7 tasks for the candidates to perform.  Note however that the tests where held individually from each candidate, but they were all giving the same set of tasks. 
During the tests, candidates where only given one tasks at a time to focus on - using small tasks cards.

The candidates where assigned the following tasks. The questions is designed so that the candidates makes use of all the features in policy engine:

\begin{framed}
To save energy the heating throughout the building is automatically turned of around 17:00. However this Wednesday around 19:00 - 22:00 an exclusive presentation is held in room number. 5 on 1. floor.
You are asked to maintain a temperature at 21 degrees throughout the presentation.

\end{framed}


\begin{framed}
It is summertime and the overall temperature inside the building is rising. You are asked to keep temperature at maximum 22 in all of the rooms in the  building.
\end{framed}


\begin{framed}
All afternoon between 12:00 and 16:00 the sun is at its peak. Therefore you are asked to set blinds ON in all the rooms on 1. floor and 2. floor - but only if the lights are ON.
\end{framed}

After the think aloud test, we made changes to front-end and redid the test with the same tasks to verify the improvements. 
After the second round of testing we felt comfortable with the usability for the purpose of this project. However there still remains some point where the front-end could be further improved.

\textbf{(Note: Insert results from think aloud test and write what we have learned from this)}

\section{Technology}
In the beginning of the project we talked about the many technologies available for web development. Obviously there are many approaches to the same solution. Focussing on working as a team, that all have different background experiences we decided to individually write down our development qualifications.

From all of the qualifications in the group, we joined together and made a top three technology approached (see Tabel \ref{tbl:dev_environment}. Every team member voted 1, 2 and 3 on an development approach. 1 being the most preferred, and three the less preferred. 

\begin{table}[h]
\caption{Team members preferred development approaches}\label{tbl:dev_environment}
    \begin{tabular}{rccccc}
    Dev environment        & Kasper & Thomas & Stefan & Rasmus & Nicolas \\
    PHP, JavaScript, MySQL & 2      & 1      & 3      & 3      & 1       \\
    C\#.Net, MS SQL         & 3      & 3      & 2      & 1      & 3       \\
    Java, JSP, Tomcat      & 1      & 2      & 1      & 2      & 2       \\
    \end{tabular}
\end{table}

One of the challenges in this project was technology readiness. Every team member has a preferred development language, which ultimately led to a long debate on settling with a development language, and what kind of framework the group should use.
The problem was that: either way - some team members would have a harder time participating with implementation than others.

Also some members had already lots of experience in development web applications, while other had very little. Not everyone was familiar with MVC based development, which for those unfamiliar became a challenge. The overall problem was simply having different level of knowledge regarding web development and technologies.

Ultimately however the group decided to go with Java as the main programming language, along with Javascript for front-end parts.

Various milestone for the project was planned and the different implementation tasks, was assigned to different team members. Most noticeable the team unintentionally created subgroups: divided in team members working with front-end tasks, and team members working with backend tasks.
However, frequent meetings and online conversations led to positive awareness on what everyone was working with.

Furthermore the team utilize an online project management site, from which the team members could keep each other informed of work in progress.

Also Github was used to easily share and version code and documentation between team members. This made it possible for everyone to see changes and follow ongoing development.
