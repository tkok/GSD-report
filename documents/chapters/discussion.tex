%\section{Discussion}\label{sec:discussion}
In this section we are going to discuss our project in three different parts, one being the collaboration with the team members from Kenya, the other being the product and lastly the overall project. We have chosen to analyse each of these areas with an approach of highlighting what could have been done differently and perhaps in a better and more successful way. We will reflect on our choices made during the project and finally also on our learning outcomes.

\section{Collaboration}\label{sec:discussioncollaboration}
The collaboration between the students at ITU, Denmark and Strathmore, Kenya, proved to be a rather challenging task. Only one person from Strathmore showed interest in the project, even though this should have been four persons. We believe that this is not caused by our approach towards the students from Kenya but instead by a lack of interest and willingness to work on the project.
With that being said, there could have been multiple ways we could have achieved a better performing team. Our collaboration strategy is still in the early phases due to the fact that we as a group never got to create any real collaboration platform to work from. 
A next approach, that we have tried multiple times to do, could be to gather every team member for an online Skype meeting. This way we would be able to, during a discussion of the current project, to develop multiple different subgroups, in the team, and not just the two general subgroups as Denmark and Kenya. With these subgroups in place, one would have a much better foundation and a more solid team.

\section{Product}\label{sec:product}
(It is difficult to discuss the product when it is not done)
In the section product, we want to discuss our final product. Questions one could ask:

\begin{itemize}
	%%%% ASLAK: What works?
	\item What could be improved?
	\item Prototype? %%% Explain that it is a prototype
	\item Future work?
\end{itemize}


\section{Project}\label{subsec:project}
In this section we want to discuss the overall project and not only focus on the end product. Instead we want to put the collaborative experience in the spotlight and focus on what could be improved regarding the collaboration, such as tools and methods. 

It has become clear to us that we could not have a lot of things differently in regards to the collaboration. This has become clear because we believe that the lack of interest in the project from the Kenyan group does not originate from our approaches but instead from a direct lack of interest in the project. Two members have directly responded that they do not wish to participate in the project, one is not actively participating and the last group member is highly unreliable. These motivational challenges are core values that one needs to have. No strategies or collaborative approaches could have transformed the situation to something positive. These challenges are not a part of a normal work setting, thus one probably would be reprimanded by not attending meetings, replying to emails and show lack of motivation.

If the initial motivation was there, a set of different methods could have been used to kick off a great collaborative team. Some of them are discussed in \ref{chapter:method}. One approach that could have been useful was to introduce each others countries and talk about topics not directly related to the core project. This way one would learn the different team members to know, and know more about them than their professional skills. This could have been done by just simply introducing one self, family life, interests, hobbies and the likes.

We suffered a lot from this during our especially first iteration. Our hopes and expectations for the collaboration was high, and thus we wanted our Kenyan team members to give us feedback and inputs on our early choices of programming platform, setup and the likes. We probably waited too long time for their opinions, which caused a delay in our project schedule. We could have solved this by creating a group contract that one had to obey, and if not, one was not part of the group any more. We could have used more strict deadlines and structured the work even more. This way we would have an early indication of their motivation, or lack hereof, and would be able to push the project forward without their involvement. 

We decided to assign different switchable roles to each person in the group. The roles switched during the project, but our focus was to achieve the highest performance throughout the process. The roles were assigned through an analysis of each persons set of skills, motivation and interest. Some persons focused more on the backend, while others worked on the frontend, and others on the report, as their primary tasks. This has showed to be a useful tactic because of two different reasons: First of all we know that the assigned person is motivated to work with the assigned area and secondly has the skills, or wants to learn how, to solve the problem at hand.

Another useful tactic was our weekly meetings, with a simplistic agenda: What is the current status and how can we push the project forward? All of our communication during these meetings were very specific and all related to these challenges and how to solve them. 

One thing that could have meant a lot to us and our project would have been, if it was possible for us to directly contact the assigned Kenyan teacher, and asked him to contact the students and have them explain to him what hinders them to participate. We also believe that it could have made a change if the Universities coordinators agreed upon a specific kick off date, where the team members would meet online and coordinate the work from here. This would ensure that we at least got to meet all of our team members. 

%%%%%ASLAK: Here you have three audiences: 1. Next years GSD Students (give advice). 2. Your future self (reflect to learn). 3. The people arranging next years GSD (more reflection).




