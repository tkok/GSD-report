
%ASLAK could go away 
In this section we are going to discuss our project in three different parts:

\begin{itemize}
	\item The collaboration with the team members from Kenya
	\item The prototype system
	\item The overall project
\end{itemize}

We have chosen to analyze each of these areas with an approach of highlighting what could have been done differently and perhaps in a better and more successful way. We will reflect on our choices made during the project and finally also on our learning outcomes.


\section{Collaboration}\label{sec:discussioncollaboration}
The collaboration proved to be a rather challenging task. Only one person from Strathmore showed interest in the project, even though this should have been four persons. We believe that this is not caused by our approach towards the students from Kenya but either; 1) a lack of interest and willingness to work on the project or 2) bad luck with the student selection - they might be swamped in work or having too many demanding courses.

We believe we did all we could in regards to motivating our foreign colleagues. In the beggining we sent them an e-mail to introduce ourselvs, based on the TA's experience regarding past collaborations with students from Kenya. We asked them in a kind way to update our "contact information" document, so we could reach each other on Skype/chat/email etc. The last member from the Kenyan group updated the contact information after 16 days - and thereafter was not heard from again. We created a mailing list (gsdall@netstarter.dk) with all our email addresses to ease the communication between all the members. All the emails were written in english, even when collaborating in the danish group. We wrote out status messages, at least once every week. During the entire project, we only heard from two of the kenyans. One of them counted for almost all communication. We also tried to schedule Skype meetings, on our weekly full working day (tuesdays). However, only one kenyan was there, and he was only online around half the tuesdays. And often it was very late (16-20:00) until he came online, while the danish team members had been at it since 10:00 AM (12:00 PM in Kenya). Perhaps this would have worked better if we would have switched time for the meetings, but unfortunately this was not possible due to time overlapping with other courses.

In order to improve our collaboration, it might have been possible to create sub-groups, ie. one from Kenya and one from Denmark - and then assigned those groups a task. These subgroups would have been useful for the implementation part, where different tasks would have been assigned to subgroups, making it easier to communicate and deliver. Because the fact that only once in a while there were two members of the kenyan group online, we decided to make two subgroups in our team, one working for the front-end and one on the back-end, while keeping informed the kenyan group of the progress.  When we could begin to see the deadline approaching, we only involved that one active guy from Kenya in some GUI developments. The others would have been invited to participate also, clearly, but they never came online or answered any of our 30+ status emails. At one point, we tried to discover what are the issues in the Kenyan group for not joining in the project and we were told that either they do not have time or it is hard to get online. Also, at this moment one of the members said that it would not participate at all due to personal reasons. 

\section{Prototype system}\label{sec:product}
Our prototype system has implemented the requirements to communicate with the building's server and execute policies. The user interface allows the possibility of creating, deleting, updating a policy and visualizing all policies. Details about one policy can be seen, and updates can be made. 
We have supplied several JUnit test classes to create policies we find relevant. When policies are being made and persistent, they \textit{are} executed against the live building simulator. Therefore we believe that the system, as a prototype system, is working sufficiently well.
Though we have tried to make the policy engine as flexible as possible, some issues remain. For example, if we are having a policy that rolls down the blinds during the night, they should roll up when the policy is no more enforced to run (decided by the toTime property). This is not possible at the moment just by having one policy. We can of course create a new policy that rolls up the blinds at a certain time in the morning, but this is not the best way to handle this issue.

\subsection{Future Work \& Improvements}\label{subsec:improvements}
As explained earlier, our expression language should be improved to allow multiple expressions with the choice of OR as opposed to the current AND-only functionality. With the GPL's on the market today, people expect to be able to make complex conditions - even though the users of this system are not IT experts. 

There should also be made some enhancements to the SetStatement functionality. We have discussed several functionalities that would be nice to have, but the two most important ones are \textit{Time} and \textit{State}. Explained shortly, \textit{Time} can be used in two ways. It should be able to conditionally query if the current time is within some specified range. This could be used for making policies behave differently on certain times (for example, from 14:00 to 19:00 on Wednesdays). Time can be added relatively simple, as a basic-edition wrapper of the GregorianCalendar class. \textit{States} are thought to be boolean variables accessible globally and locally - dependent on which scope it was declared. This way, for example, a fire detection policy could set the \textit{buildingIsOnFire} state, and each other policy would be able to change it's behavior if needed based on access to that state. States can also be programmed easily, by having a map or a hashtable with key and boolean value pairs. The local states can be implemented by using the same structure for the global states, but with the policy id as a prefix in the name.

The reason we did not develop the above mentioned functionalities was the time pressure.

\section{Project}\label{subsec:project}
In this section we want to discuss the overall project and not only focus on the end product. Instead we want to put the collaborative experience in the spotlight and focus on what could be improved regarding the collaboration, such as tools and methods. 

It has become clear to us that we could not have a lot of things differently in regards to the collaboration. This has become clear because we believe that the lack of interest in the project from the Kenyan group does not originate from our approaches. Two members have directly responded that they do not wish to participate in the project, one is not actively participating and the last group member is highly unreliable. These motivational challenges are core values that one needs to have. No strategies or collaborative approaches could have transformed the situation to something positive. These challenges are not a part of a normal work setting, thus one probably would be reprimanded by not attending meetings, replying to emails and show lack of motivation. 

If the initial motivation was there, a set of different methods could have been used to kick off a great collaborative team. Some of them are discussed in \ref{chapter:method}. One approach that could have been useful was to introduce each others countries and talk about topics not directly related to the core project. We actually tried opening up for this kind of communication early in the project, by making a "Swahili - Danish" dictionary document. The active kenyan team member helped filling out different words, and the idea was then that we should try to speak a little in each others language when Skyping. Unfortunately, due to the general lack online kenyan team members, we never really used it. The few times we actually communicated with them via voice, it would have been too 'forced' and strained to use that approach. We would have liked to communicate more and more often with our kenyan team members. This way one would learn the different team members to know, and know more about them than their professional skills. This could have been done by just simply introducing one self, family life, interests, hobbies and the likes.

We suffered a lot from this during our especially first iteration. Our hopes and expectations for the collaboration was high, and thus we wanted our Kenyan team members to give us feedback and inputs on our early choices of programming platform, setup and the likes. We probably waited too long time for their opinions, which caused a delay in our project schedule. We could have solved this by creating a group contract that one had to obey, and if not, one was not part of the group any more. This was discussed during the danish team members, but it would have been a ultimatum - put forth mainly due to frustration and not to improve the collaboration. We could have used more strict deadlines and structured the work even more. This way we would have an early indication of their motivation, or lack hereof, and would be able to push the project forward without their involvement. 

We decided to assign different switchable roles to each person in the group. The roles switched during the project, but our focus was to achieve the highest performance throughout the process. The roles were assigned through an analysis of each persons set of skills, motivation and interest. Some persons focused more on the backend, while others worked on the frontend, and others on the report, as their primary tasks. This has showed to be a useful tactic because of two different reasons: First of all we know that the assigned person is motivated to work with the assigned area and secondly has the skills, or wants to learn how, to solve the problem at hand. This should not be interpreted as a person was 'stuck' doing only one thing. Team members could easily shift to other areas, and that did also happen.

Another useful tactic was our weekly meetings, with a simplistic agenda: What is the current status and how can we push the project forward? All of our communication during these meetings were very specific and all related to these challenges and how to solve them. 

It should have been possible for us to directly contact the assigned Kenyan teacher, and asked him to contact the students and have them explain to him what hinders them to participate. If ITU are going to do a collaborative project with Strathmore University again, we would advice that ITU implements much tighter teacher to teacher communication - resembling employer to employer conversations held in private companies when venturing into a partnership. We believe it should have a consequence that a 5 man team effectively is only a unreliable 1 man team. We expect that the Global Software Development should somewhat resemble real world setting. We find it unrealistic implementing such an idea without any leverage mechanism. 

We also believe that it could have made a change if the Universities coordinators agreed upon a specific kick off date, where the team members would meet online and coordinate the work from here. This would ensure that we at least got to meet all of our team members. 