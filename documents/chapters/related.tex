%\section{Related work}
\label{chap:relatedwork}
A lot of research has emerged lately in the field of energy management systems for smart buildings. 
A similar work with the one presented in this paper has been done by Tiberkak et. al \cite{Tiberkak10}, where a Policy Based Architecture for Home Automation
System is developed. The system is composed of the following subsystems: one responsible
for home automation, one for the local management of rooms, one for storing data, and a system for enabling communication between the first two sub systems. Different profiles are defined to improve energy efficiency.
The concept of preferred and required profile is introduced, to differentiate between the preferences of a inhabitant and the maximum and minimum values of each environmental parameter in the required profile. Notifications
and messages are sent between the layers when there is a change in the status of a room, and a appropriate decision is taken.

Another approach is the one taken by Li et. al \cite{Li11} where they implement a energy management system for homes, that provides automatic metering and capability of taking decisions based on monitoring energy consumption. 
Tasks can be used to specify different actions required at different moments. A simulation has been done for 1000 homes and by using their system, they achieved a significant energy reduction.

In \cite{Han10} a complete system for home management is described. Using ZigBee and IEEE 802.15.4, it is easy to add new devices, connect them with already existing ones and control them using a remote device. However, no kind of automatic management exists for the light service, heating and air conditioning. In \cite{Wen11} develop a prototyping platform based on Building Controls Virtual Test Bed framework \cite{Bcvtb} for controlling and testing networked sensors and actuators on physical system. An algorithm for controlling the light and blind system in the room has been developed. The system is configured to provide an illumination of 500 lux between 6 AM and 6 PM. This has been achieved by measuring the daylight and setting the blind to block direct sun beam into the room. The system managed to achieve a reduction by an average of 17\% of cooling demands and a maximum of 57\% of lighting demand compared to the reference room.

Krioukov et al \cite{Krioukov12} take another approach with their Building Stack Application. They provide an application programming interface and runtime for portable building applications. Developers can express their intent in a easy way e.g. " turn off the lights
for top floor cubicles near windows" or can handle building differences by having support for programatic exploration of the building's components. It consists of three layers, the query based API layer, the driver layer and the abstract interconnection layer. 
The query layer provides functionality for selecting objects based on attributes, type and functional or spatial relationships. The driver layer is composed of high-level and low-level drivers to encompass the number of different components in a given building. The low-level defines the types of the components and the high-level provides descriptive information of those components. The asbstract interconnection layer is a RESTful interface to facilitate communication between the application's request and the building control protocols. 
