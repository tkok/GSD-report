\section{Related work}
A lot of research has emerged lately in the field of energy management systems for smart buildings. 
One work that has similarities with this paper has been done by Tiberkak et. al \cite{Tiberkak10}, where a Policy Based Architecture for Home Automation
System is developed. The system is composed of five subsystems: one responsible
for home automation, one for the local management of rooms, one
for storing data, and a system for enabling communication between the first
two sub systems. Different profiles are defined to improve energy efficiency.
The concept of preferred and required profile is instroduced, to differentiate
between the preferences of a inhabitant and the maximum and minimum values
of each environmental parameter in the required profile. Notifications
and messages are sent between the layers when there is a change in the
status of a room, and a appropriate decision is taken.
Another approach is the one taken by Li et. al \cite{Li11} where they implement a energy management system for homes, that provides automatic metering and capability of taking decisions based on monitoring energy consumption. 
Tasks can be used to specify different actions required at different moments. A simulation has been done for 1000 homes and by using their system, they achieved a significant energy reduction.
