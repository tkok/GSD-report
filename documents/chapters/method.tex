\section{Method}
%\label{sec:method}
Before any actual work could start, one preliminary goal was to figure out how we could make our group work together as one. Actually this challenge is even more challenging in this project than in a normal work situation: No organization is in order, no predefined roles, no actual project goals and the likes. This chapter will focus on these challenges and how we tried to handle these. We will highlight different methods to create social interaction and understanding. We will focus on how one can rationalize collaboration. Afterwards we will discuss the different tools we used throughout the project life cycle with collaboration in mind. Finally, we will discuss how one can manage a virtual project.\\

\subsection{Social Context}
When we discuss the \textit{Social Context}, we discuss the direct milieu in which the person is and how different factors can influence this person. Communication is also a part of the social context, which is not necessarily only between two persons but can be between one to many persons, in different time zones, different cultures etc.\\
\paragraph{Common ground}
Our first step to connect to our fellow group mates in Kenya was to introduce ourselves via an e-mail and just shortly highlight some common information about each person from ITU, like stating name, age etc. This method is known as creating ``common ground'', as introduced by Olson and Olson \textbf{REFERENCE}. The term to create common ground ``refers to that knowledge that the participants have in common, and they are aware that they have it in common''\textbf{REFERENCE}. Common ground is not only established through simple general knowledge about each participant. It is also created through a persons behaviour and appearance through meetings and conversations. We tried to use this method as a way of getting to know our team members, to create a level of understanding and finally to create a stepping stone from which the project could evolve from.

\paragraph{Trust and First Impression}
This initial contact was already quite frustrating because of the fact that it was difficult to get a reply from some of the group members in Kenya, only two members was relatively easy to get in touch with. This leads directly to two different considerations in our group work: ``Trust''textbf{REFERENCE} and ``First impressions matters''textbf{REFERENCE}. Trust in group work is a value of high much the different team members trust in each other. How much does one believe that the other team members will deliver their part of the necessary work? How much does one believe that a mail we be answered? How well does the team work together? The trust is between the two subgroups, Denmark and Kenya, relatively low because of the amount -and lack of- replies and general communication. At the time being we only expect one member from Kenya to be online during our team sessions but at the same time we expect everybody from the ITU-group to be online at every session. This is also translatable from the first impressions that we received from the group from Kenya. It is not in any way rewarding for the group atmosphere not to join group conversations and not replying emails. 

\paragraph{Collaboration Readiness}
The literature for these challenges seems to agree that these sort of problems generally arise from two different topics. One being ``Collaboration- and Technology Readiness'' and the other ``Continuities/Discontinuities''. The latter part will be discussed in the section below, \ref{Virtual Project Management}.

Collaboration readiness is a potential show stopper for the team work, if a given member is not ready to collaborate. This could be caused by having conflicts in interest, e.g. one is about to overtake another persons job or the likes. This could cause that the person, who is about to lose his job, would not be ready to collaborate. We have tried to identify these issues towards our fellow group members and we cannot find anything that should indicate that they would not be willing to collaborate. They should be just as interested in delivering a good product. 

One thing that could cause their lack of interaction in our e-mail correspondences and Skype meetings are the technology readiness. We know that it is a challenge for some of the group members to get internet access because of the fact that not all of them have a connection in their home. Our approach to solve this issue was to have a meeting each Tuesday at 10:00. This way we know that they should have access to their University, which most of the time has an internet access they could use. We know they should have time for this meeting, thus it is planned as a course on their schedule.

\paragraph{Ethnocentrism}
A potential threat to the well-being and harmony of the group is known as Ethnocentrism. Ethnocentrism is a state of one subgroup, where the members sees that one group as the centre of everything, and every other group will be valued and ranked from this. A subgroup is a group inside a group, some of the group members are part of - but not the whole group is part of. This way it could create some sense of ``Us versus Them'', which is something you definitely want to avoid. One way to avoid this is to create multiple subgroups inside the group. One subgroup could be of everybody that likes football and know that they had this in common. If you are part of multiple subgroups the feeling of being part of just one group will dissolve, which should result in a more harmonious group.
This way of creating subgroups would be something that you did early on during the initial communication, and something we tried to solve by writing small parts about each other member from Denmark. Unfortunately, we did not receive any feedback from Kenya. This has probably strengthened the feeling of us vs. them, because we do not know much about them. We definitely have a feeling that our group is the centre right now due to the fact that it is only the Danish group that develop and contribute to the project.

\paragraph{Coupling of work}
Coupling of work relates to the state of the current tasks and how loosely or tightly coupled they are. A completely loosely coupled work is one you can perform without the interaction and feedback from other persons, this could for instance be some of the work done at a large factory. A tightly coupled work task is, on the other hand, one you only can perform with other members of the group participating. Our project has evolved from a very tightly coupled project to a more loosely coupled. This is done on purpose. In the beginning of the project everybody had to be at the meetings because of the fact that we had to define which way to move the project. The development life cycle was quite rigid and strict. After the initial phase, where we decided on the platform, chose an architecture etc., more and more tasks became slightly more loosely coupled. This means that one could start to work on his part of the project without any direct interaction with other team members. This would also allow such a highly distributed team as our to collaborate in an efficient way. It would just be to inefficient if everybody had to be together at the same time and place every time anything regarding the project should happen. As of right now we communicate through different Groupware Tools (see \ref{Groupware Technologies}) and only meet through one weekly meeting.


TODO:
\begin{itemize}
  \item Common ground - done
  \item Coupling of work (loosely coupled is the goal. Highly coupled, everything has to be changed, everything is working together, complex tasks, you want the opposite) - done
  \item Trust - done
	\item Ethnocentrism (Sub Groups)
	\item Collaboration readiness - done
	\item Technology readiness - done
	\item First impression matters -done
\end{itemize}

\subsection{Collaborative Work}
Collaborative work across cultures is a challenge. In our case we had to work together 9 people with 4 being from Kenya, a culture and country that we before entering this project, didn't know much about. As mentioned earlier the social context and the process of creating 'common ground' with the collaborators is of high importance. The goal is to create a fundamental shared understanding of the task and build up the motivation and very much needed trust for the collaboration to succeed. Cooperative work is defined by Schmidt REFERENCE as ''People engage in cooperative work when they are mutually dependent in their work and therefore are required to cooperate in order to get the work done,'' (Schmidt)

The bigger the group the more 'articulation work', articulation work is the extra activities required for collaboration. The task at hand defines what is the actual work and what is articulation work. Articulation work is about who does what, when and where. There are mechanisms of interaction that supports the process when articulation work cannot be handled through every day social interaction. These mechanisms are for instance: Organizational structures (formal/informal), plans, schedules and conceptual schemes. What all these mechanisms have in common is that they all strive to reduce the effort in labor, resources, time, etc. required to handle articulation work. 
Our strategy for the articulation work was to define processes and choose the groupware technologies that supported our cause best possible. 

Increased ''reach'' of a task changes the coordination. The more spread out between people and artifacts a task is the more the reach is increased. Segregation is a suitable strategy when a complex task is at hand. By dividing the complex task into smaller tasks you get to simultaneously solve them individually and thereby complete the larger complex task faster. It also allows for more specialised teams to investigate a task at a more detailed level. Therefor we segregated the tasks within our project into smaller more comprehensible tasks. Group members were assigned to these tasks that would work closer together until the task was completed. 

To handle these tasks we created a project plan with deadlines and milestones to keep track of everything. Moreover we agreed on having a status meeting every week, where we would discuss progress, issues, ideas etc. Arranging a meeting where all is able to attend is not always easy. We did manage to 'meet' on Skype at a time which took into consideration both the differences in time zones (temporal discontinuities) and the fact that people had entirely different classes and work schedules.

With computer supported cooperative work (CSCW), it's impossible to anticipate every contingency which might occur, there will always be exception handling. The core challenges and dimensions of cooperative work includes articulation work, adaptation of technologies and awareness. The lack of trust and awareness when you never meet face to face with your collaborators is problematic and requires methods and training when using communication tools. We primarily used Skype to communicate with and made sure to document changes and commits to the codebase very detailed. Individuals working together need to be able to gain some level of shared knowledge about each other's activities.


\begin{itemize}
  \item Cooperative work
  \item Articulation work
  \item Awareness
	\item Coordination of actions
	\item Management of coupling
	\item Coordination of mechanisms
\end{itemize}

\subsection{Groupware Technologies}
\begin{itemize}
  \item Adoption process
  \item Adaptation
  \item Critical mass
\end{itemize}

\subsection{Virtual Project Management}
\begin{itemize}
  \item Continuities
  \item Discontinuities
  \item Virtual meetings
	\item Virtual team dynamics
\end{itemize}

->START NOTES

-> Discontinuities
'gaps or lack of coherence in aspects of work, such as work 
setting, task, and relations with other managers' (Watson-Manheim et al., 2002:193)

'a discontinuity, edge, or other dividing characteristic 
present in the work context of team' (Espinosa et al.2003:158)


General properties of discontinuities 
Not stable, but changeable 
Can be seen as highly interlinked – continuum
Can emerge and change over time as people adapt in the teams 
Discontinuities may only affects parts of the work 
Discontinuities often appear in bundles (for instance geography+time
+organizational)
Examples of discontinuities: Temporal (working across time zones, 
geographic work location, work group membership( e.g. who you work 
with), organizational affiliation and cultural backgrounds, expertise related 
(novice vs experts), historical (different version of a product), different 
professions (e.g. developers and researchers) or different technologies.

-> Continuities

Stable factors in the collaboration that the participants are 
aware of and consciously act on, or they may be implicit and 
unrecognized. (Watson-Manheim et al., 2002:200) 

Continuities can appear to routine or invisible in order to 
overcome discontinuities
Can be described as strategies or factors to overcome 
discontinuities

-> END NOTES
