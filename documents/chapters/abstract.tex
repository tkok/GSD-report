This paper presents the process of the development of a policy engine platform. The team behind the platform consists of members from respectively Strathmore University, Kenya and IT University, Denmark. 

The motivation for doing this project is to dive into the world of distributed collaboration between to teams in different countries, while working on an real-life project. The field of building automation has seen a lot of growth in recent years. Building automation can save a lot of energy and resources by e.g. turning off light when nobody is in the room and the likes. This could possibly be a massive improvement in countries such as Kenya where there is a lack of resources. It would allow the resources to be used in the right conditions and not just wasted.

The general approach to the project is to solve the aforementioned policy project while collaborating with team members across continents. First steps is to create a fundamental team platform to communicate from. Afterwards we will focus on the technical project.

Our result is a fully working prototype of a policy engine platform, where one can create, edit and de-/activate policies. The engine is validated through a series of test, including functionality- and user tests. We show that our approach towards the development of the platform is both valid and appropriate in relation to the goals, requirements and audience.