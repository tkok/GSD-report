This paper presents the process of the development of a policy engine platform, used in the domain of building automation to regulate the internal environment of a building. The team behind the platform consists of members from respectively Strathmore University, Kenya and IT University, Denmark. 

The motivation is to dive into the world of distributed collaboration between two teams in different countries, Denmark and Kenya, while working on a real-life problem. The field of building automation has seen a lot of growth in recent years. Building automation can save energy and resources by e.g. turning off light or regulating the temperature when nobody is in the room. Conservation of energy and natural resources through building automation offers wast improvements compared to buildings where this automation is not deployed.

The general approach to the project is to solve the aforementioned policy engine project while collaborating with team members across continents. The first step was to create a fundamental team communication platform. Afterwards we focused on the technical project, where both the danish and the kenyan collaborated as one group with multiple subgroups in regards to people's preferences and interests.

The end result was a fully working prototype of a policy engine platform, where a user can create, edit and de-/activate policies. The engine was validated through a series of tests, including low level functionality tests and hight level user tests. We argue that our approach towards the development of the platform is both valid and appropriate in relation to the goals, requirements and audience.