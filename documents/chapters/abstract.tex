This paper presents the process of the development of a policy engine platform. The team behind the platform consists of members from respectively Strathmore University, Kenya and IT University, Denmark. 

The motivation in regards to the project is to dive into the world of distributed collaboration between two teams in different countries, while working on a real-life project. The field of building automation has seen a lot of growth in recent years. Building automation can save a lot of energy and resources by e.g. turning off light when nobody is in the room. This could possibly be a massive improvement in countries such as Kenya where resources of such kind is lacking. It would allow the usage of these resources to give the most value and not by going to waste.

The general approach to the project is to solve the aforementioned policy project while collaborating with team members across continents. First step is to create a fundamental team platform to communicate from. Afterwards we will focus on the technical project, where both parties will collaborate as one group with multiple subgroups in regards to people's preferences and interests.

Our end result is a fully working prototype of a policy engine platform, where a user can create, edit and de-/activate policies. The engine is validated through a series of tests, including functionality and user tests. We show that our approach towards the development of the platform is both valid and appropriate in relation to the goals, requirements and audience.