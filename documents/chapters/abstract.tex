This paper presents a distributed, collaborative software prototype development project in the domain of building automation. The prototype system --- the policy engine -- is used for regulating the internal environment of a building. The global team behind the prototype consists of members from Strathmore University, Kenya and the IT-University of Copenhagen, Denmark. 

The field of building automation has seen a lot of growth in recent years. Building automation can save energy and resources by e.g. turning off light or regulating the temperature when nobody is in the room. Conservation of energy and natural resources through building automation offers wast improvements compared to buildings where this automation is not deployed. 

The approach to the project is to analyze, design and implement the policy engine prototype in a distributed, collaborative project. First we created the communication platform for the project. Then we managed the project while simultaneously developing the prototype. During the length of the project we founded multiple subgroups in regards to team members preferences and interests - seeking to optimize both our quality and quantity of code.

The end result was a fully working prototype of a policy engine platform, offering functionality to create, edit and execute policies. The engine was validated through a series of tests, including low level functionality tests --- unit tests --- and hight level user tests. 

We argue that our approach towards the development of the prototype is both valid and appropriate in relation to the goals, requirements and audience.