In this project we presented our solution of a web-based platform to manage the scheduling of governing policies for a building. This was done by reading and controlling its sensors and actuators. The project group consisted of members from Strathmore University in Nairobi, Kenya and IT University in Copenhagen, Denmark.

The general challenge of the project was to develop a prototype of a policy engine, while collaborating in a distributed team. 

We presented our functional prototype which is able to create, modify and delete policies. The prototype executes the policies against a provided building simulator. We have designed the prototype with usability and flexibility in mind, and it aims to be used by people that are not IT experts, but building administrators. This resulted in having a simple and user friendly web interface that is manageable by the end users. 

Another goal of this project was the collaboration with team members from Kenya. This collaboration has not been a success for several reasons. Mainly that the Kenyans did not respond. Not even to simple emails. Every initial step of collaboration have been ignored by the Kenyan team members. A tendency that sadly kept on throughout the entire project development.

We as the group from ITU, and the teaching staff (Supervisor and Teaching Assistants) - could have done things differently as described in section~\ref{sec:discussioncollaboration}. But ultimately, the lack of commitment from Kenya is the reason that the global collaboration failed.

Despite the fact that the entire software development was done by only five people from Denmark; And that the project was assigned to nine including the Kenyans - the prototype of the Policy Engine works as we have expected. However there is always room for improvements as described in section~\ref{subsec:improvements}. For instance we would have liked to implement temporal constraints in the Statement mechanism, this would have provided even more flexibility when defining policies. Also the possibility to use OR in the expressions is something that we feel is lacking in the current version.

The usability testing proved to be a valuable method to validate the user interface, it pointed out areas that needed improvement that we then could improve to make a better solution.