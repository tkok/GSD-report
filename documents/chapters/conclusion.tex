In this project we present our solution of a web-based platform to manage the scheduling of governing policies for a building. This is done by reading and controlling its sensors and actuators. The project group consists of members from Strathmore University in Nairobi, Kenya and IT University in Copenhagen, Denmark.

The general challenge of the project was to develop a prototype of a policy engine, while collaborating in a distributed team. 

We presented our functional prototype which is able to create, modify and delete policies. The prototype executes the policies against a provided building simulator. We have designed the prototype with usability and flexibility in mind, and it aims to be used by people that are not IT experts, but building administrators. This resulted in having a simple and user friendly web interface that is manageable by the end users. 

Another goal of this project was the collaboration with team members from Kenya. This collaboration has not been a success for several reasons. Mainly that the Kenyans did not respond. Not even to simple emails. Every initial step of collaboration have been ignored by the Kenyan team members. We as the group from ITU, and the teaching staff (Supervisor and Teaching Assistants) - could have done things differently but, ultimately, the lack of commitment from Kenya is the reason that the global collaboration failed.

The prototype of the Policy Engine works as we have expected - even though it has room for some improvements. One feature that we would like to have implemented is temporal constraints in the Statement mechanism. This would provide even more flexibility when defining policies.


%%The improvements should not be introduced here. Are they a part of the future work section?

%%%Not the right place: 
% We thank Matias Bjørling, Javier González and Aslak Johansen for their feedback and offered support on the building simulator. 

