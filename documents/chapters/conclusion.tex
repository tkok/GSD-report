In this project we present our solution of a web-based platform to manage the scheduling of governing policies for a building. This is done by reading and controlling its sensors and actuators. The project group consists of members from Strathmore University in Nairobi, Kenya and IT University in Copenhagen, Denmark.

The general challenge of the project was to develop a prototype of a policy engine, while collaborating in a distributed team. 

We presented our functional prototype which is able to create, modify and delete policies. The prototype executes the policies against the provided building simulator. We have designed the prototype with usability and flexibility in mind, and it aims to be used by people that are not IT experts. This resulted in having a simple and user friendly interface that is manageable by the end users. 

Another goal of this project was the collaboration with team members from Kenya. The collaboration has not been a success for several reasons. The Kenyans did not respond to even simple emails and every initial step of collaboration was ignored. We, and the teaching team, could have done things differently but in the end, the lack of commitment from Kenya resulted in a failed effort of collaboration.
 
The prototype of the policy engine works as expected, even though it has room for several improvements. One feature that we would like to have is temporal constraints in the Statement mechanism. This would provide even more flexibility in defining policies. Another improvement would be to employ an analysis mechanism of running policies, which would detect possible conflicts (setting different values on the same actuators) between running policies.
%%The improvements should not be introduced here. Are they a part of the future work section?

%%%Not the right place: 
% We thank Matias Bjørling, Javier González and Aslak Johansen for their feedback and offered support on the building simulator. 

