\section{Collaboration Evaluation}\label{sec:discussioncollaboration}
In this section we discuss the overall project and the collaboration details. For a technical discussion see \ref{chapter:design} and \ref{chapter:implementation}. 

The collaboration proved to be a rather challenging task. Only one person from Strathmore showed any interest in the project, even though our initial expectation was a four man team. We believe that the lack of interest was not caused by our approach towards the students from Kenya but either; 1) a lack of interest and willingness to work on the project or 2) bad luck with the student selection - they might have been swamped in work or having too many demanding courses.

We believe we did all we could in regards to motivating our foreign colleagues. In the beginning we sent them an e-mail to introduce ourselves, based on the TA's experience regarding past collaborations with students from Kenya. We asked them in a kind way to update our ``contact information'' document, so we could reach each other on Skype/chat/email etc. The last member from the Kenyan group updated the contact information after 16 days - and thereafter was not heard from again. We created a mailing list (gsdall@netstarter.dk) with all our email addresses to ease the communication between all the members. All the emails were written in English, even when communicating between the Danish team members. We wrote out status messages, at least once every week. During the length of the project, we only heard from two of the Kenyans. One of them counted for almost all communication, and he willingly became the \textit{team leader} for the Kenyans --- though that did not help on their work efforts. We also tried to schedule Skype meetings, on our weekly full working day (Tuesdays). However, only one Kenyan was there, and he was only online around half the Tuesdays. And often it was very late (16-20:00) until he came online, while the Danish team members had been at it since 10:00 AM (12:00 PM in Kenya). Perhaps this would have worked better if we would have switched time for the meetings, but unfortunately this was not possible due to overlapping with other courses.

In order to improve our collaboration, we would have liked to create sub-groups of participants mixed from Kenya and Denmark, and assigning those groups a task. However, we never managed to set it up, due to the lack of commitment from the Kenyans. These subgroups would have been useful for the implementation part, where different tasks would have been assigned to subgroups, making it easier to communicate and deliver. Instead we decided to make two subgroups in the Danish team, one working on the front-end and one on the back-end, while keeping the Kenyan group informed about our progress, and motivating them to write about their efforts. Approximately one month before deadline, we only focused on involving the one active Kenyan in some GUI developments. The others would have been invited to participate also, clearly, but they never came online or answered any of our 30+ status emails. Through the elected Kenyan team leader we were told that either they did not have the time to participate or that it was too hard to get online. One of the members also quitted the course, but we were unaware of this situation until we asked their leader.

It has become clear to us that we could not have done much differently in regards to the collaboration. Two members have directly responded that they do not wish to participate in the project, one is not actively participating and the last group member is highly unreliable. These motivational challenges are core values that one needs to have to participate in a global software development project. No strategies or collaborative approaches could have transformed the situation into something positive. These challenges are not a part of a normal work setting, thus one probably would be reprimanded by not attending meetings, for not replying to emails and for not showing any motivation towards the work.

If the initial motivation was there, a set of different methods could have been used to kick off a great collaborative team. Some of them are discussed in \ref{chapter:method}. One approach that could have been useful was to introduce each others countries and talk about topics not directly related to the core project. We tried opening up for this kind of communication early in the project, by asking the one active Kenyan guy to help us make a dictionary between our two languages. He explained there are many languages, but Swahili is the official language. As a result we made a "Swahili - Danish" dictionary document. The active Kenyan team member helped filling out different words, and the idea was then that we should try to speak a little in each others language when Skyping. Unfortunately, due to the general lack online Kenyan team members, we never really used it. The few times we actually communicated with them via voice, it would have been too ``forced'' and strained to use that approach. We would have liked to communicate more and more often with our Kenyan team members. This way one would get to know the different team members and know more about them. This could have been done by just simply introducing one self, family life, interests, hobbies and the likes.

If the initial motivation was there, a set of different methods could have been used to kick off a great collaborative team. Some of them are discussed in \ref{chapter:method}. One approach that could have been useful was to introduce each others countries and talk about topics not directly related to the core project. We tried opening up for this kind of communication early in the project, by asking the one active kenyan guy to help us make a dictionary between our two languages. He explained there are many languages, but Swahili is the official language. As a result we made a "Swahili - Danish" dictionary document. The active kenyan team member helped filling out different words, and the idea was then that we should try to speak a little in each others language when using Skype. Unfortunately, we never really used it as the kenyan group members were not online for meetings. The few times we actually communicated with them via voice, it would have been too ``forced'' and strained to use that approach. We would have liked to communicate more and more often with our kenyan team members, this way we could have gotten to know the different team members and know more about them. This could have been done by just simply introducing one self, family life, interests, hobbies and the likes.

We suffered a lot from the lack of contact with the Kenyans, especially during the first iteration. Our hopes and expectations for the collaboration were high, and thus we wanted our Kenyan team members to give us feedback and inputs on our early choices of programming platform, setup and other decisions. We probably waited too long time for their opinions, which caused a delay in our project schedule. We could have solved this by creating a group contract that one had to obey, and if not, one was not part of the group any more. This was discussed during a meeting by the Danish team members, but it would have been a ultimatum - put forth mainly due to frustration and not to improve the collaboration. We could have used more strict deadlines and structured the work even more. This way we would have an early indication of their motivation, or lack hereof, and would be able to push the project forward without their involvement. 

It should have been possible for us to directly contact the assigned Kenyan teacher, and asked him to contact the students and have them explain to him what hindered their participation. We have been informed that the intermediaries between ITU and Strathmore University were the TAs\footnote{Teaching Assistants} and the course manager. We tried several times to address this issue, but either the intermediaries did nothing to elevate the issue at Strathmore University or, more likely, it did not have any effect. If ITU is going to do a collaborative project with Strathmore University again, we would advice that ITU implements much tighter teacher to teacher communication - resembling employer to employer conversations held in private companies when venturing into a partnership. We believe that there should be some consequences because from a 4 man team only one actually showed interest, and still was highly unreliable. We expect that the Global Software Development should somewhat resemble real world setting. We find it unrealistic implementing such an idea without any leverage mechanism.

We also believe that it could have been a good idea if the Universities coordinators would have agreed upon a specific kick off date, where the team members would have met online and coordinate the work from here. This would ensure that we at least got to meet all of our team members, and had some basic knowledge about each other. We strongly hope that this would be taken in consideration for the next time when the course would run.

We decided to assign different switchable roles to each person in the Danish group. The Kenyan team leader was at one point also assigned a role as he showed interest in working on the GUI development. The roles switched during the project, but our focus was to achieve the highest performance throughout the process. The roles were assigned through an analysis of each person's set of skills, motivation and interest. Some persons focused more on the backend, while others worked on the frontend, and others on the report, as their primary tasks. This has showed to be a useful tactic because of two different reasons; a) first of all we know that the assigned person is motivated to work with the assigned area and secondly b) he has the skills, or wants to learn the skills, to solve the problem at hand. This should not be interpreted as a person was ``stuck'' doing only one thing. Team members could easily shift to other areas, and that did also happen.

Another useful tactic was our weekly meetings, with a simplistic agenda: What is the current status and how can we push the project forward? All of our communication during these meetings were very specific and all related to these challenges and how to solve them. The benefit was that everybody was up to date with the current progress, knew the challenges and had the possibility to participate in finding a solution. These statuses were also emailed through our group email, so if people could not attend, there was the possibility to catch up.
